\documentclass[11pt]{article} 
\setlength{\topmargin}{-0.7in}
\setlength{\oddsidemargin}{-0.25in}
\setlength{\evensidemargin}{-0.25in}
\setlength{\textwidth}{7in}
\setlength{\textheight}{9.5in}
\usepackage{graphicx}

\begin{document}

\title {Homework 7: LaTex Programing}
\author{Lan Zou, 1030032}
\maketitle


Before, we did mathematic programing in this way
\begin{figure}[h]
 \begin{center}
  \includegraphics[width=3.5in]{Calculus.JPG}
 \end{center}
\end{figure}

Right now, we are doing programming by LaTex
\begin{figure}[h]
 \begin{center}
  \includegraphics[width=4in]{Programing.JPG}
 \end{center}
\end{figure}

\section{LaTex}
"LaTeX is a document markup language and document preparation system for the TeX typesetting program. The term LaTeX refers only to the language in which documents are written, not to the editor used to write those documents." 

\label{fig:Calculus}
\ref{fig:Calculus}
Wikipedia

\newpage
Below is my favourite mathematics equation
in dealing with
geometric question
\[
9^2 + 12^2 = 15^2
\]
This comes from equation
\[
3^2 + 4^2 = 5^2
\]which can be written as many different equations, but holds the same mathematics relationship. 
\[
\]
Here is a table contained more detail information about Homework 1 of Stat 302.
\begin{center}
 \begin{tabular}{| l | c | r |}
  \hline
    Student Number: 1030032 & Name: Lan Zou & 2013/5/21 \\ \hline
    \hline
    *Question & *Answer & *Line Number \\ \hline
  College or University & University of Washington & 1 \\ \hline
 	Department & Mathemathics & 2\\ \hline
 	Course & Math 480 & 3\\ \hline
 	Class Room & Low 101 & 4 \\ \hline
  \hline
 \end{tabular}

\end{center}

\begin{enumerate}
\item Top Line: Indicates the student's identity and the date finished 
\item Question: The first column is what we looking for
\item Answer: The second column is the information we collected
\item Line Number: The third column provides how many questions we asked as well as the answers we got 
\end{enumerate}

\end{document}
